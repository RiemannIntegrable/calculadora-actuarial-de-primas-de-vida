%% ==== CONFIGURACIÓN DE IDIOMA Y CODIFICACIÓN ====
\usepackage[T1]{fontenc}
\usepackage[spanish]{babel}
\spanishdecimal{.}

%% ==== CONFIGURACIÓN DE PÁGINA ====
\usepackage[papersize={216mm, 279mm},tmargin=20mm,bmargin=20mm,lmargin=20mm,rmargin=20mm]{geometry}
\renewcommand{\familydefault}{\sfdefault}
\setlength{\parindent}{0mm}

%% ==== ENCABEZADO Y PIE DE PÁGINA ====
\usepackage{fancyhdr, lastpage}
\pagestyle{fancy}
\fancyhf{}
\lhead{Universidad Nacional De Colombia}
\rhead{Facultad De Ciencias}
\cfoot{\thepage\ de \pageref{LastPage}}
\lfoot{Departamento De Matemáticas}
\rfoot{Contingencias de vida}
\renewcommand{\headrulewidth}{0.08pt}
\renewcommand{\footrulewidth}{0.08pt}

%% ==== PAQUETES MATEMÁTICOS ====
\usepackage{
  amsmath, 
  amssymb, 
  amsfonts, 
  yhmath, 
  amsthm, 
  mathtools, 
  mathrsfs, 
  cancel, 
  bigints, 
  fixmath
}
\usepackage{makecell}
\providecommand{\norm}[1]{\lVert#1\rVert}

%% ==== PAQUETES PARA BIBLIOGRAFÍA ====
\usepackage[style=ieee]{biblatex}
\addbibresource{referencias.bib}

%% ==== PAQUETES PARA GRÁFICOS Y DIAGRAMAS ====
\usepackage{
  graphicx,
  tikz,
  tikz-cd,
  pgfplots,
  venndiagram,
  xcolor
}
\pgfplotsset{compat=1.18}
\usepgfplotslibrary{external}
\tikzexternalize
\usepackage[all]{xy}

%% ==== PAQUETES PARA FORMATO Y ESTRUCTURA ====
\usepackage{
  lipsum, 
  multicol, 
  float, 
  multirow, 
  array, 
  tcolorbox, 
  anyfontsize, 
  xltxtra
}

%% ==== PAQUETES PARA CÓDIGO Y BIBLIOGRAFÍA ====
\usepackage{listings}
\usepackage{hyperref}
\usepackage{dirtree}
\usepackage{fontspec}

%% ==== DEFINICIÓN DE ENTORNOS MATEMÁTICOS ====
\newtheorem{theorem}{Teorema}[section]
\newtheorem{lemma}[theorem]{Lema}
\newtheorem{proposition}[theorem]{Proposición}
\newtheorem{corollary}[theorem]{Corolario}

\theoremstyle{definition}
\newtheorem{definition}{Definición}[section]
\newtheorem{remark}[definition]{Observación}

\theoremstyle{remark}
\newtheorem{example}{Ejemplo}[section]
\newtheorem{exercise}{Ejercicio}
\newtheorem{question}{Pregunta}
\newtheorem{answer}[question]{Respuesta}
\newtheorem{solution}[exercise]{Solución}

\renewcommand\qedsymbol{$\blacksquare$}