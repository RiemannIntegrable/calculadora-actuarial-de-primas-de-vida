Queremos caracterizar la prima de un seguro fraccionario con el siguiente patrón de valor asegurado:

\begin{figure}[H]
    \centering
    \includegraphics[scale=0.3]{../images/Screenshot 2025-06-04 140629.png}
\end{figure}

Sabemos que eso es simplemente la suma de todas las primas de los seguros fraccionarios temporales diferidos multiplicadas por el valor asegurado de ese momento, es decir: 

\begin{equation*}
    (I_rA^{(m)})_x = \sum_{k=0}^\infty (1 + kr) {}_{k|}A^{(m)}_{\overset{1}{x}:\angl{1}}
\end{equation*}

Para poder realizar los cálculos utilizando la tabla de mortalidad vamos a asumir UDD, por lo que tenemos:

\begin{align*}
    \sum_{k=0}^\infty (1 + kr) {}_{k|}A^{(m)}_{\overset{1}{x}:\angl{1}} &= \sum_{k=0}^\infty (1 + kr) \cdot \frac{i}{i^{(m)}} \cdot {}_{k|}A_{\overset{1}{x}:\angl{1}} \\
    &= \frac{i}{i^{(m)}} \sum_{k=0}^\infty (1 + kr) {}_{k|}A_{\overset{1}{x}:\angl{1}} & \text{(Usamos UDD)}\\
    &= \frac{i}{i^{(m)}} \left[ \sum_{k=0}^\infty {}_{k|}A_{\overset{1}{x}:\angl{1}} + r \sum_{k=0}^\infty k \cdot {}_{k|}A_{\overset{1}{x}:\angl{1}} \right] \\
\end{align*}

\begin{align*}
    &= \frac{i}{i^{(m)}} \left[ \sum_{k=0}^\infty v^k \cdot {}_{k}p_x \cdot q_{x+k} + r \sum_{k=0}^\infty k \cdot v^k \cdot {}_{k}p_x \cdot q_{x+k} \right] \\
    &= \frac{i}{i^{(m)}} \left[ A_x + r \sum_{k=0}^\infty k \cdot v^k \cdot {}_{k}p_x \cdot q_{x+k} \right] \\
    &= \frac{i}{i^{(m)}} \left[ A_x + r \left( \sum_{k=0}^\infty (k+1) v^k \cdot {}_{k}p_x \cdot q_{x+k} - \sum_{k=0}^\infty v^k \cdot {}_{k}p_x \cdot q_{x+k} \right) \right] \\
    &= \frac{i}{i^{(m)}} \left[ A_x + r \left( (IA)_x - A_x \right) \right] \\
    &= \frac{i}{i^{(m)}} [A_x + r((IA)_x - A_x)]
\end{align*}

Esto es lo que esta programado en \texttt{punto1.R}. Para ejecutar la función el único requisito es que la tabla de mortalidad sea un dataframe de \texttt{R} y que tenga al menos las columnas \texttt{x} y \texttt{qx}. La función no requiere una edad mínima en la tabla porque detecta la edad mínima y máxima y sobre eso realiza los cálculos. La función también tiene una pequeña validación del tipo de los datos antes de realizar los cálculos.