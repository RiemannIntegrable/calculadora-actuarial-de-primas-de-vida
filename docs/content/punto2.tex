Queremos caracterizar la prima de un seguro fraccionario con el siguiente patrón de valor asegurado creciente dentro del año:

\begin{figure}[H]
    \centering
    \includegraphics[scale=0.3]{../images/Screenshot 2025-06-04 135626.png}
\end{figure}

Sabemos que es la esperanza del pago descontado a valor presente, es decir: 

\begin{equation*}
    (I^{(m)}_rA^{(m)})_x = \sum_{k=0}^\infty \sum_{j=0}^{m-1} \left(1+r\left(k+\frac{j}{m}\right)\right) v^{k+\frac{j+1}{m}} {}_{k + \frac{j}{m}|\frac{1}{m}} q_x
\end{equation*}

Para poder realizar los cálculos utilizando la tabla de mortalidad vamos a asumir UDD. El desarrollo de esto esta en la siguiente demostración y la implementación esta en \texttt{punto2.R} y tiene los mismos requerimientos que la función anterior.